\documentclass[journal]{combine}

% !TeX spellcheck = de_DE 
\usepackage{csquotes}

\renewcommand{\contentsname}{Inhaltsverzeichnis}

\begin{document}
	
	% Define document title and author
	\title{SWEN1 Praktikum 1 - Artefakte}
	\author{ R. Monteiro Simoes}
	\maketitle

	\newpage

	\tableofcontents

	\newpage
	
	\section{Offerte im Projekt erstellen - \emph{fully dressed}}
	
	\begin{description}
		\item[Scope:] \hfill  \\Offerttool
		\item[Level:] \hfill \\User goal
		\item[Primary Actor:] \hfill \\Verkäufer
		\item[Stakeholders and Interests:]
		\begin{description}]
			\item[Verkäufer:]\hfill \\Möchte effiziente, fehlerfreie Eingabefunktionalität 
			für die Offertenerstellung.
			\item[Firma:]\hfill \\Möchte das die Offerteneinträge gemäss 
			NPK und Firmenangebot erstellt wird.
			\item[Endkunde:]\hfill \\Möchte das die erhaltene Offerte 
			übersichtlich ist und gemäss NPK.
			\item[Mitarbeiter:]\hfill \\Möchte eine detailierte Auflistung der
			offerierten Arbeit und des Materials.  
		\end{description}
		\textbf{Precondition:} Verkäufer ist eingeloggt, Projekt existiert. \\
		\textbf{Success Guarantee :} Offerte ist gespeichert. Beträge
		werden richtig kalkuliert. Offerte kann als PDF ausgedruckt werden. \\
		\textbf{Main Success Scenario:} 
		\begin{enumerate}
			\item 
		\end{enumerate}
	\end{description}
	
	\newpage
	\section{Arbeits- / Materialprotokoll  - \emph{casual}}
		\emph{Satndardszenario:} \\
		Der Ausführende Arbeiter akzeptiert die vorgeschlagenen Aufwände für Material
		und Zeit. Das System speichert die Einträge ab.\\
		\emph{Alternatives Szenario I:}\\
		Der Ausführende passt die vorgeschlagenen Aufwände für Material und Zeit an.
		Das System speichert diese angepassten Einträge separat ab.
		\emph{Alternatives Szenario II:}\\
		Der Ausführende möchte nicht aufgelistete Aufwände selber hinzufügen. Er kann neue
		Einträge für Zeit- und Materialaufwände hinzufügen. Das System speichert diese separat ab.
	\newpage
	\section{Schlussrechnung erstellen - \emph{brief}}
		Der Nutzer öffnet die Offerte, die Material- und Arbeitsaufwände hat,
		die zu einer Schlussrechnung werden soll. Die Applikation führt den User durch
		unstimmigkeiten und fordert den User auf, einen Input vorzunehmen. Das System
		speichert allfällige änderungen ab. Der User kann die bearbeitung abschliessen.
		Das System markiert entsprechend die Offerte als final. Der User kann eine Datei der 
		Schlussrechnung laden.


\end{document}