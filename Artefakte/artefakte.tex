\documentclass[journal]{combine}


\usepackage{tikz,ifthen,xstring,calc,pgfkeys,pgfopts}
\usepackage{tikz-uml}

\renewcommand{\contentsname}{Inhaltsverzeichnis}

\begin{document}
	
	% Define document title and author
	\title{SWEN1 Praktikum 1 - Artefakte}
	\author{ R. Monteiro Simoes}
	\maketitle

	\newpage

	\tableofcontents

	\newpage
	
	\section{Offerte im Projekt erstellen - \emph{fully dressed}}
	
	\begin{description}
		\item[Scope:] \hfill  \\Offerttool
		\item[Level:] \hfill \\User goal
		\item[Primary Actor:] \hfill \\Verkäufer
		\item[Stakeholders and Interests:] \hfill 
		\begin{description}
			\item[Verkäufer:]\hfill \\Möchte effiziente, fehlerfreie Eingabefunktionalität 
			für die Offertenerstellung.
			\item[Firma:]\hfill \\Möchte das die Offerteneinträge gemäss 
			NPK und Firmenangebot erstellt wird.
			\item[Endkunde:]\hfill \\Möchte das die erhaltene Offerte 
			übersichtlich ist und gemäss NPK.
			\item[Mitarbeiter:]\hfill \\Möchte eine detaillierte Auflistung der
			offerierten Arbeit und des Materials.  
		\end{description}
		\item[Precondition:] \hfill \\Verkäufer ist eingeloggt, Projekt existiert. 
		\item[Success Guarantee :] \hfill \\Offerte ist gespeichert. Beträge
		werden richtig kalkuliert. Offerte kann als PDF ausgedruckt werden. 
		\item[Main Success Scenario:] \hfill  
		\begin{enumerate}
			\item Der Verkäufer fügt dem angewählten Projekt eine neue Offerte hinzu.
			\item Das System startet den Prozess für die erstellung einer neuen Offerte und
			verlangt vom Verkäufer einen Titel sowie eine Auswahl aus angegebenen Baubranchen
			\item Der Verkäufer gibt einen Titel sowie die entsprechende Baubranche an
			\item Das System erstellt und speichert eine neue Offerte mit dem 
				angegebenen Titel
			\item Der Verkäufer fügt dem Projekt einen Standardisierten Artikel hinzu
			\item Das System informiert den Verkäufer über die existierenden Artikel
			\item Der Verkäufer wählt ein Artikel aus, gibt die Stückzahl an sowie 
			denn allfälligen Rabatt
			\item Das System fügt der Offerte den Artikel hinzu mitsamt allen angegebenen Daten\\
			\emph{Die Schritte 5 - 8 können beliebig oft wiederholt werden}
			\item Der Verkäufer fügt dem Projekt einen manuellen Artikel hinzu
			\item Das System fragt den Verkäufer, ob er einen existierenden manuellen Artikel hinzufügen 
			oder einen neuen erstellen möchte
			\item Der Verkäufer möchte einen existierenden manuellen Artikel hinzufügen
			\item Das System informiert den Verkäufer über existierende Manuelle Artikel
			\item Der Verkäufer wählt einen manuellen Artikel aus, gibt die Stückzahl an sowie 
			denn allfälligen Rabatt
			\item Das System fügt der Offerte den Artikel hinzu mitsamt allen angegebenen Daten\\
			\emph{Die Schritte 9 - 14 können beliebig oft wiederholt werden}
			\item Der Verkäufer schliesst die Offertenerstellung abschliessen
			\item Das System speicher die Daten und berechnet ein Total für die Offerte
			\item Der Verkäufer bekommt die Offerte als Datei
		\end{enumerate}
		\newpage
		\item[Extensions:] \hfill  
		\begin{enumerate}
			\item [4.a] Das System meldet bei falscher Formatierung, fehlenden Daten
			oder Speicherproblemen dies dem Verkäufer und geht zu Punkt \emph{2} zurück.
			\item [8.a] Das System meldet bei falscher Formatierung, fehlenden Daten
			oder Speicherproblemen dies dem Verkäufer und geht zu Punkt \emph{5} zurück
			\item [11.a] Der Verkäufer möchte einen neuen manuellen Artikel hinzufügen
			\begin{enumerate}
				\item[1.] Das System startet den Prozess für die erstellung eines neuen manuellen Artikels
				und informiert den Verkäufer über die möglichen Typen eines manuellen Artikels
				\item[2.a] Der Verkäufer wählt den Typen \emph{Text}
				\begin{enumerate}
					\item[1.] Das System verlangt vom Verkäufer einen Text
					\item[2.] Der Verkäufer gibt den Text an
				\end{enumerate}
				\item[2.b] Der Verkäufer wählt den Typen \emph{Produkt}
				\begin{enumerate}
					\item[1.] Das System verlangt vom Verkäufer einen Namen, eine Beschreibung,
					das Mengenformat sowie einen Preis
					\item[2.] Der Verkäufer gibt einen Namen, eine Beschreibung, den Mengenformat
					sowie den Preis an
				\end{enumerate} 
				\item[3.] Das System speichert die angegebenen Angaben als neuen manuellen Artikel
				und fügt diesen der Offerte hinzu.
				\begin{enumerate}
					\item [1.] Das System meldet bei falscher Formatierung, fehlenden Daten
					oder Speicherproblemen dies dem Verkäufer und geht zu Punkt \emph{11.a.1} zurück
				\end{enumerate} 
			\end{enumerate}
			\item [14.a] Das System meldet bei falscher Formatierung, fehlenden Daten
			oder Speicherproblemen dies dem Verkäufer und geht zu Punkt \emph{9} zurück
			\item [16.a] Das System meldet bei Speicher-  oder Verbindungsproblemen
			dies dem Verkäufer und speichert eine lokale Kopie der Dateien und geht danach
			zu Punkt \emph{16} zurück.
		\end{enumerate}
	\end{description}
	
	\newpage
	\section{Arbeits- / Materialprotokoll  - \emph{casual}}
		\emph{Standardszenario:} \\
		Der Ausführende Arbeiter akzeptiert die vorgeschlagenen Aufwände für Material
		und Zeit. Das System speichert die Einträge ab.\\
		\emph{Alternatives Szenario I:}\\
		Der Ausführende passt die vorgeschlagenen Aufwände für Material und Zeit an.
		Das System speichert diese angepassten Einträge separat ab.\\	
		\emph{Alternatives Szenario II:}\\
		Der Ausführende möchte nicht aufgelistete Aufwände selber hinzufügen. Er kann neue
		Einträge für Zeit- und Materialaufwände hinzufügen. Das System speichert diese separat ab.
	\newpage
	\section{Schlussrechnung erstellen - \emph{brief}}
		Der Nutzer öffnet die Offerte, die Material- und Arbeitsaufwände hat,
		die zu einer Schlussrechnung werden soll. Die Applikation führt den Verkäufer durch
		unstimmigkeiten und fordert ihn auf, eine Eingabe vorzunehmen. Das System
		speichert allfällige Änderungen ab. Der Verkäufer kann die bearbeitung abschliessen.
		Das System markiert danach die Offerte als final. Der Verkäufer kann eine Datei der 
		Schlussrechnung laden.

	\newpage
	\section{Anwendungsfalldiagramm 1}
	\begin{tikzpicture}
		\umlactor[x=-1]{A}
		\umlactor[x=1]{B} 
		\umlassoc{A}{B}
	 \end{tikzpicture}


\end{document}