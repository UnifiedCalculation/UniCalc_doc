\documentclass[journal]{combine}

% !TeX spellcheck = de_DE 
\usepackage{csquotes}

\renewcommand{\contentsname}{Inhaltsverzeichnis}

\begin{document}
	
	% Define document title and author
	\title{SWEN1 Praktikum 1 - Artefakte}
	\author{ R. Monteiro Simoes}
	\maketitle

	\newpage

	\tableofcontents

	\newpage
	
	\section{Offerte im Projekt erstellen - \emph{fully dressed}}
		Hier wird der use case 1 in einer grossen Detailtreue beschrieben
	\newpage
	\section{Arbeits- / Materialprotokoll  - \emph{casual}}
		\emph{Satndardszenario:} \\
		Der Ausführende Arbeiter akzeptiert die vorgeschlagenen Aufwände für Material
		und Zeit. Das System speichert die Einträge ab.\\
		\emph{Alternatives Szenario I:}\\
		Der Ausführende passt die vorgeschlagenen Aufwände für Material und Zeit an.
		Das System speichert diese angepassten Einträge separat ab.
		\emph{Alternatives Szenario II:}\\
		Der Ausführende möchte
	\newpage
	\section{Schlussrechnung erstellen - \emph{brief}}
		Der Nutzer öffnet die Offerte, die Material- und Arbeitsaufwände hat,
		die zu einer Schlussrechnung werden soll. Die Applikation führt den User durch
		unstimmigkeiten und fordert den User auf, einen Input vorzunehmen. Das System
		speichert allfällige änderungen ab. Der User kann die bearbeitung abschliessen.
		Das System markiert entsprechend die Offerte als final. Der User kann eine Datei der 
		Schlussrechnung laden.


\end{document}